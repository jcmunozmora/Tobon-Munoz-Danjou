\documentclass[12pt,a4paper,english]{article}%

\usepackage{amssymb}
\usepackage{amsfonts}
\usepackage{amsmath}
\usepackage{mathrsfs}
\usepackage{threeparttable}
\usepackage[nohead]{geometry}
\usepackage[singlespacing]{setspace}
\usepackage[bottom]{footmisc}
\usepackage{indentfirst}
\usepackage{float}
\usepackage{endnotes}
\usepackage{graphicx}%
\usepackage{rotating}
\usepackage{amsfonts}
\usepackage{placeins}
\usepackage{authblk}


\usepackage[colorlinks=true,citecolor=black,linkcolor=black,urlcolor=black]{hyperref}
\usepackage{multimedia}
\usepackage{multirow}
\usepackage{rotating}
\usepackage{booktabs}
\usepackage{tikz}


\usepackage[nohead]{geometry}
\usepackage[singlespacing]{setspace}
\usepackage[bottom]{footmisc}
\usepackage{indentfirst}
\usepackage{endnotes}
\usepackage{amssymb}

\usepackage{caption}
\usepackage{appendix}
\usepackage{longtable}
\usepackage{rotating}
\usepackage{epigraph}
\usepackage{appendix}
\usepackage{lscape}
\usepackage{subfig}
\usepackage{framed}
%%%%%% REFERENCES %%%

\renewenvironment{thebibliography}[1]
{\section*{\refname}%
\@mkboth{\MakeUppercase\refname}{\MakeUppercase\refname}
\list{\@biblabel{\@arabic\c@enumiv}}
	{\settowidth\labelwidth{\@biblabel{#1}}
	\leftmargin\labelwidth
	\advance\leftmargin20pt
	\advance\leftmargin\labelsep
	\setlength\itemindent{-20pt}
	\@openbib@code
	\usecounter{enumiv}
	\let\p@enumiv\@empty
	\renewcommand\theenumiv{\@arabic\c@enumiv}}
\sloppy
\clubpenalty4000
\@clubpenalty \clubpenalty
\widowpenalty4000
\sfcode`\.\@m}
{\def\@noitemerr
	{\@latex@warning{Empty `thebibliography' environment}}%
	\endlist}
\renewcommand\newblock{\hskip .11em\@plus.33em\@minus.07em}
\makeatother
\makeatletter
\def\@biblabel#1{\hspace*{-\labelsep}}
\makeatother
\geometry{left=1in,right=1in,top=1.00in,bottom=1.0in} 

%%%%%



\usepackage{apacite}
\usepackage[colorlinks=true,citecolor=black,linkcolor=black,urlcolor=black]{hyperref}
\usepackage{epigraph}

\captionsetup{font={normalsize,bf}}
\graphicspath{{tables/}}

\usepackage{endnotes}

\usepackage[english]{babel}
%\addto\captionsenglish{
%\def\figurename{Map}
%}

\newcommand{\goestable}[1]{\begin{center}[Table \ref{table_#1} goes about here]\end{center}}
\newcommand{\goesmap}[1]{\begin{center}[Map \ref{map_#1} goes about here]\end{center}}
\renewcommand{\thefootnote}{\fnsymbol{footnote}}
\begin{document}


\begin{center} 
\textbf{{\Huge The institutional logic behind illicit behavior:} \\ \Large Informal property rights and the expansion of coca crops in Colombia\footnote{We gratefully acknowledge funding from the European Union Seventh Framework Programme (FP7/2007-2013) under grant agreement no: 263905 (TAMNEAC).} }\\[0.5cm]

{\large Santiago Tob\'on-Zapata \footnote{ Universit\'e Catholique de Louvain. Department of Economics, Coll\'ege L. H. Dupriez, 3 Place Montesquieu B- 1348 Louvain-la-Neuve. Email: \textit{santiagotobon@gmail.com}} \\ \normalsize \textit{Universit\'e Catholique de Louvain}}\\[0.5cm]

{\large Juan Carlos Mu\~noz-Mora\footnote{European Center for Advanced Research in Economics and Statistics -ECARES . Universit\'e Libre de Bruxelles. 50, Avenue Roosevelt CP 139, 1050 Brussels, Belgium  e-mail:\textit{juan.carlos.munoz.mora@ulb.ac.be}.}  \\ \normalsize \textit{ECARES - Universit\'e Libre de Bruxelles}}\\[0.5cm]

{\large Jesse Willem d'Anjou\footnote{Center for European Research in Microfinance, Universit\'e Libre de Bruxelles. 50, Avenue Roosevelt CP 114, 1050 Brussels, Belgium, e-mail:\textit{jesse.w.d.anjou@gmail.com}.}  \\ \normalsize \textit{CERMI - Universit\'e Libre de Bruxelles}}\\[1.5cm]
{\large Draft Version }\\
\large \today 
\end{center}

\textbf{Abstract} % ORGANIZAR %.
\begin{footnotesize} 
Property rights as crucial institutions for development have long been the focus of academic attention, from a variety of perspectives. However, the effects of the level of formalization of property rights on illegal activities have so far remained absent in the literature. Considering the large scale of the cocaine production market in Colombia and the vast amount of money invested to counter this market in the war on drugs, property rights could be a crucial element in the explanation why certain land plots are utilized to grow coca leafs, and others for other purposes. This study focuses on this relationship, combining the presence of illicit coca leaf plantations with certain levels of property rights in Colombia. Furthermore, it is the first study to disentangle these effects in relation to the consistent violence around illicit crops in Colombia. Taking the specific characteristics of the coca cultivation process into account, we use a dynamic model in combination with the system GMM estimator which allows us to control for individual specific effects and persistent variables, and to introduce lags of endogenous variables as instruments. Additionally, we perform a Granger causality test to establish the correct direction of this relationship. The empirical data stems from unique census dataset covering 892 municipalities in Colombia from 2000 to 2008. Our results suggest that weaker institutional structures of property rights over land have a positive effect on the number of coca leaf plantations. These findings seem to suggest that efforts in the war on drugs should also be focused towards the formalization of property rights. 
\end{footnotesize}
\\[0.5cm]
\textbf{Keywords:} Institutions; Property rights; Illicit crops; War on drugs; System GMM. \\
\textbf{JEL Classification Numbers:} C23, D74, O17, P14, P37, Q15.

\pagebreak%

\onehalfspacing
\renewcommand{\thefootnote}{\arabic{footnote}}
\addtocounter{footnote}{-3}


\section{Introduction}
\label{intro}

A renewed interest has arisen since the publication of Acemoglu and Robinson�s book Why Nations Fail towards the role of institutions as crucial elements in the success or failure of a society, thereby explaining phenomena such as growth, inequality, and poverty \cite{Acemoglu:2001tk,Acemoglu:2005ua,Acemoglu:2010ve,Acemoglu:2012wb,Acemoglu:1998uk,Besley:1995ti,Besley:2010wr,North:1981vm,Ostrom:2009wp,Rodrik:1999uz} As part of the broader concept of (economic) institutions, defined as the framework and rules under which human interaction develops in societies \cite{North:1981vm}, property rights also embody a social function and its exercise yields further limitations as they are designed to fulfill collective services \cite{Duguit:1912uj,Reich:1964uf}. Property rights play an important part in the story on the role of institutions, through which constructive behavior, but also illicit behavior both on the government side, as well as the citizen side could get incentivized (i.e. illicit activities such as illegal logging, or drug production). 

Property rights fall in two general categories: formal and informal. Notwithstanding the origins of this economic institution in Colombia at this day, formal and informal property rights appear however to be coexisting next to each other in the country, although clustering in certain regions. Considering the history of drugs production and the related violence in Colombia, this paper studies the possible relationship between formal or informal property rights on the one hand, and illicit coca crop production and the other, whereby the connection with the apparent violence is introduced in the models as well. Although these concepts have been empirically studied individually and in pairs, the three have never been combined to disentangle these relationships. Furthermore, since Why Nations Fail the renewed discussion has focused on the direction of the relationships between institutions and certain behavior. These relationships are the particular focus of this study, whereby insights are supplied as to the role of a certain institution in the presence of illicit coca crop fields and violence in Colombia.    
     
Colombia has been plagued for decades by illicit drug production, violence, guerilla and the opposing paramilitary groups, and instable governments with their war on drugs.  The cocaine production and exporting market has played a central role in this. The demand and resulting profits for this product are enormous. The global cocaine market in 2009 was estimated in US\$ 89 billion (2011 US dollars), with 14 to 20.5 million yearly users \cite{UNODC:2011tf}. Before 1995, most of the coca fields to produce the raw paste were located in Peru and Bolivia, with the processing plants in Colombia. Since then, a shift of these fields has been observed to neighboring Colombia. This was due to two factors:  first, the flight interdiction program implemented by the government of Peru with the help of the government of the United States resulted in large decrease in coca paste supply to Colombia. Secondly, with the fall of communism in the 90s, the financial support for the left-wing guerillas in Colombia dissolved \cite{Thoumi:2002wb}. Both factors resulted in the cross-border relocation of coca production fields close to the processing fields in Colombia. By 2010, the growth of the coca production fields in Colombia decreased considerably again to 1995 levels, due to the implementation of counter narcotics programs of the Colombian government.

The production process to produce the end product cocaine takes place in four stages: planting, growing and harvesting of coca leafs, extraction of coca paste, transformation into cocaine base, and finally the conversion of cocaine base into cocaine hydrochloride. The process starts in regions of Bolivia, Colombia and Peru where biological and environmental conditions are optimal for growing the plant and moreover, a culture of coca farming has profound roots in indigenous communities. Most species and varieties of coca plants grow below 1,500 meters, they are perennial, harvested on average four times per year and reach full maturity between 12 and 24 months after sowing the seeds  \cite{Bray:1983td,GAO:2008uo,Hanna:1974wd,Mejia:2010tx,Matteucci:1996vt,MorenoSanchez:2003jc,Riley:1993ta}. The War On Drugs of the Colombian government against the cocaine industry takes place throughout the whole process. Some of the main policies include seizure of raw materials for production, manual and aerial eradication of coca fields, destruction of production plants and laboratories, interdiction of drug shipments, as well as promotion of alternative development and crop substitution programs\cite{Mejia:2010tx}. Most of these policies are encouraged and financed jointly by producing and consuming countries. An example of this was Plan Colombia, which originated in 1998 as a bilateral cooperation program with an annual spending of US\$ 1.7 billion (2011 US dollars) between the governments of Colombia and the United States to fight against illegal drugs and organized crime \cite{DNP:2006ux}. Although not without any success, these programmes have proven to be limited when it comes to decreasing the number and area of coca crop plantations, as well as disturbing the finances and influence of the organized crime behind it. Recent studies favor alternatives that focus on increasing households' income from legal activities. These alternatives include technical assistance, agricultural loans and more secure forms of property rights \cite{Grossman:2008wh,Ibanez:2010tr,Mejia-Restrepo_2008,MorenoSanchez:2003jc,GAO:2008uo}.

The scientific foundation for the direct relationship between the promotion of property rights and decreasing the discussion to grow and produce coca crops on an individual�s land plot has remained absent so far. Property rights have been studied in relation to many economic outcomes. Findings show that improvements in property rights can boost wages, productivity, and open up the possibilities for individuals to invest, with in turn results in potential higher income of the rural households \cite{Besley:1995ti,Besley:2000ty,Deininger:2008uq,Deininger:2006ue,Acemoglu:2012wb,Demsetz:1967uy}. The effects of weak property rights have also been subject of investigation is several studies. Findings show that this could hinder law enforcement of government bodies, and increase social tensions, which in turn could generate violence and facilitate illegal recruitment, forced displacement, land appropriation, and the development of other illegal activities in conflict areas . \cite{Andre:1998tu,Binswanger:1995td,Collier-etal_2001,Collier-Hoeffler_2004,Deiniger_2003,Deininger:2009tp,Fernandez:2010wy,Velasquez_2008}


This study acknowledges this niche by analyzing the relationship between the presence of illicit coca crops and the level of formality of the property rights over land, in order to provide a wider understanding of the role of institutions in development and the underlying mechanisms behind coca crop planting. We hypothesize that more formal structures of property rights have a negative effect on the presence of coca crop plantations, since governments are able to exert law enforcement, and households increase potential income out of legal agricultural yield while they are less willing to take part in illegal activities (CHANGE WITH RESPECT TO CHAPTER II?). Therefore, we study the determinants of the presence of illicit coca plantations focusing on the effect of the informality index of property rights over land. With asking this question, we aim to contribute to the debate on the importance of institutions in macro-economic phenomena, as well as supply policymakers in the War On Drugs with insights that could help to steer more effective policies.

The empirical data in this study stems from a unique census-dataset on sub-national level from Colombia covering the period from 2000 to 2008. This data is enriched with data on the presence of coca fields from the United Nations Office on Drugs and Crime (UNODC), which is estimated for every municipality. To measure the level of formality of the property rights, a proxy variable was devised. This proxy is based upon the safeness in land ownership as defined in Colombian legislation. More concrete, we use the informality index of property rights over land constructed by \citeA{Ibanez:2010tr} with data from the cadastral database of the Geographical Institute Agust�n Codazzi. The index is a ratio per municipality between the cadastral land area without legal title over the total cadastral area of a municipality corrected for non-private property. We acknowledge, that this index may not reflect the full sense of property rights, as it does not reflect the ability to exercise these rights. However, considering the data available, we argue that this index is a good approximation to start with.  

We propose a dynamic specification, following \citeA{Arellano:1991wr} and \citeA{Blundell:1998vr}. This is driven by several considerations. Firstly, coca plants are perennial and reach full maturity between 12 and 24 months after sowing the seeds. There is a source of persistence in the presence of illicit coca plantations as past realizations of areas with coca crops affect the current one, which could possibly cause omitted variables bias. Secondly, Colombian municipalities are heterogeneous which points at the existence of fixed individual effects. Finally, the available data shows a large number of individual units of observation, but small T. As a result, common approaches such as OLS and Fixed Effects estimators are biased and inconsistent \cite{Nickell:1981uo,Sevestre:1985wf}. Instead, we use the System GMM estimator developed by \citeA{Arellano:1995tj} and \citeA{Blundell:1998vr}. This estimator allows us to control for individual specific effects. Furthermore, it enables us to introduce lags of endogenous variables as instruments. This System GMM proves to perform better than similar estimators in the presence of persistent variables \cite{Blundell:1998vr,Blundell:2000wu}. The estimations are complemented with a wide range of geographic, political, land, and socio-economic controls which have been identified in previous, related studies. Additionally, this estimator is also used in the Granger causality test to identify the causality linkage between the presence of illicit coca plantations and the informality index of property rights \cite{Granger:1969wx,Granger:2003ts,HoltzEakin:1988wo}.

Results suggest that weaker structures of property rights have positive effects on the levels of coca leaf plantations. With all controls included, we find a positive effect of 0,849 percent in coca fields presence per 1.000 hectares, when the ratio of informal properties in a municipality increases by 1 percent. This relation is robust to several specifications, alternative measures of institutions, distinct subsamples and overidentification and serial correlation tests. Additionally, results of the Granger causality test show that higher levels of the informality index of property rights �Granger cause� an increase in the presence of coca leaf plantations, whereas the converse causality relationship is rejected. \emph{MORE RESULTS PLUS POLICY IMPLICATIONS.}

The paper is structured as follows. Section two provides the context of this study, which first puts the topic in its broader context of the role of institutions, after which the individual elements of property rights, coca crops, violence, and Colombia will be discussed through the literature. Section three outlines the identification and empirical strategy, our hypotheses, describes the data, and discusses the descriptive evidence. The econometric results, along with the Granger test results are presented in section four, which is followed by a discussion of the robustness of the findings. Section five concludes by discussing the policy implications and possibilities for future research. 

\pagenumbering{arabic}

\section{Institutions, violence and illicit crops}
\label{previous_res}

Spurred by the recent publication of the book Why Nations Fail of Baron Acemoglu and James A. Robinson, an increasing number of studies focus on the importance of institutions \citeA{Acemoglu:2012wb}. Departing from \citeA{North:1981vm}, \citeA{Jones:1987ue} , and \citeA{Engerman:2008dx} institutions such as property rights are increasingly being held as the underlying mechanism explaining micro- and macro-economic outcomes such as poverty, inequality, economic development and growth through the influence institutions exert on the behavior and incentives of individuals in real life \cite{Acemoglu:2001tk,Acemoglu:2005ua,Acemoglu:2010ve,Acemoglu:1998uk,Besley:2000ty,Coatsworth:2008tp,Besley:2010wr,North:1990vl,Ostrom:2009wp,Rodrik:1999uz}. Broadly, institutions can be divided between political and economic institutions (with informal institutions divided between these two), whereby the importance of one of the other is somewhat debated \cite{Acemoglu:2012wb}. The general argument as in \citeA{Acemoglu:2012wb} holds that political institutions have developed through history (and critical junctures), whereby the ones in power of the political institutions determine the economic institutions in a certain country or region to their benefit or to the benefit of society at large (although there is a strong synergy between the two). This eventually drives (or blocks) individuals to divert resources, invest in their future, or sit still. Although subject to large debate, it is this institutional logic that explains micro and macro differences in economic prosperity, as well as social phenomena such as poverty and inequality \cite{Acemoglu:2005ua}.
          

The economic institution of property rights play a key role in this line of literature \cite{DeSoto:2000vn,Acemoglu:2001tk,Acemoglu:2010ve,Easterly:2003vr,Hall:1999wy,Pande:2007vu}. In many rural areas in developing countries, property rights can be characterized by poor specification and weak enforcement. Furthermore, modern property rights regimes could coexist next to more traditional property right regimes in the same country, albeit in different regions \cite{Fergusson:2012tx}. Secure property rights enable individuals to invest in their land and production assets around it, innovate, and increase productivity \cite{Besley:1995ti,Johnson:2002wd,Field:2007ut,Goldstein:2008tc}. Without these rights, people are simply not willing to devote any resources to their land if they think it can be taken away from them by the government, or individuals or groups around them. However, due to the inability or unwillingness of the one�s in power, behavior of individuals which could lead to economic prosperity is blocked, through which certain others can profit \cite{Acemoglu:2011id}. In Colombia, the democratic government lacks control over the political and economic institutions in large parts of the country. This pattern of law and order is however highly scattered. In some parts, governments have firm control over law and order, being able to provide security and public services, while securing the functioning of the economic institutions. In other areas state authority is largely absent, with local left-wing guerillas and right-wing paramilitaries filling this void in their struggle over control with the accompanying violence. According to \citeA{Acemoglu:2012wb}, this tension between national politicians with power over the functional part of society on the one hand, and violence and the absence of centralized state institutions on the other, got caught in a viscous circle in Colombia. Both national politicians as well as guerilla and paramilitary groups, use to lawlessness in certain parts of the country in their own benefit.     

An increasing number of studies analyze the effects of property rights on a wide range of social and economic outcomes. Important is also how property rights are measured in these studies, since this is not as straightforward as one might think. \citeA{Besley:1995ti} looks at the effects property rights might have on land investments in Ghana. Three theoretical models are developed and empirically tested, which are based on the security of tenure, using land as collateral, and obtaining gains from trade. Clear from this study however is that property rights are subject to endogeneity as well, since farmers might invest resources to enhance their rights on the land. Notwithstanding the acknowledgement of no necessary link between the establishment of property rights and increased investment, the author finds quite supportive evidence of this relationship. However, the results also suggest more is at play in explaining the variance. Furthermore, in contrast to widely used proxy for strength in property rights constructed by the Political Risk Services (i.e. the risk of expropriation), property rights were measured through declared transfer rights of the households, which were further divided in rights to rent, sell, mortgage, pledge, bequeath and give away land. Findings suggest that this proxy did not fully capture what the household consider as important when making investment decisions. Endogeneity is also the primary argument in the study by \cite{Fergusson:2012tx} and \cite{Haber:2003wl}, in which the argument by \cite{Acemoglu:2012wb} that the political elite in charge of the economic institutions install weak property rights or enforce the property rights selectively for their own benefit.   

Other studies followed on this path by taking more economic outcomes into account. \citeA{Goldstein:2008tc} find in Ghana that the political power of individuals is correlated with more secure tenure rights, which in turn increases investment in land fertility with higher outputs as a result. Using similar measures of the transferability and security of property rights as \cite{Besley:1995ti}, \cite{Deininger:2008uq} find in Uganda that recognition of customary land laws considerably increases the already positive effects of the tenure regime and the transfer rights on investment, productivity, and land values.  Enlarging the body of evidence to land reforms, both\citeA{Besley:2000ty} and \cite{Deininger:2008uq} find a positive effect of introducing more formal property rights through land reforms on income, consumption, wages, and physical and human capital, by use of panel data.   
Although the studies above show positive effects of more secure property rights, several studies look at the opposite effects as well. In the political science literature, several studies have looked empirically at the effect of natural resources on the incidence and duration of civil wars \cite{Collier-Hoeffler_2004,Ross:2003tw}. Following the literature on civil war, \citeA{Ross:2004vy} finds that illegal resources such as drugs can increase the duration of the conflict. In Sierra Leone, anthropologist \citeA{Richards:1996wq} finds similar evidence on the roots, causes, and duration of the civil war and the fighting over the resources in the rain forest. Questioning the �New Barbarism� thesis of Robert D. Kaplan (which is related to the thesis of the �Resource Curse� of \citeA{Sachs:2001vc} ), \citeA{Richards:1996wq} argues against the idea that the roots of the civil war were to be found in the apparent social breakdown caused by population pressure and environmental collapse, and the youth grabbing the abundant natural resources of the country as a form of human greed. More so, political failures were reported by the rebel youth to be the cause, with the violence of the conflict also having its roots culturally. Of course, primitive accumulation of forest and mineral resources has fed politics. However, although not the cause of the war as in Colombia, due to several reasons the state's capacity to control some of its peripheral regions was weakened, which enabled the rebel groups to take advantage of the abundant natural resources. 

Other studies pointing in a similar direction stem from Colombia. Many of these studies focus on the relationship between certain levels of property rights, and the apparent violence in the country. \citeA{Velasquez_2008} looks into this relationship with data on the municipal level of massacre rates, attacks made by illegal armed groups and number of forced displaced people. The measure for property rights is constructed by the level of security in land ownership, as defined in Colombian legislation. This study identifies that a clear causal relationship between violence and the level of property rights in not straightforward. Through the instrumentation of lagged values, the authors do find that more secure property rights lead to a significantly lower number of attacks made by illegal armed groups and the total number of displaced people. Proving the complex causality direction between violence and property rights, \cite{Fernandez:2010wy} studies the exact opposite relationship. Using a similar proxy for property rights as \citeA{Velasquez_2008}, this author uses the instruments of past values of literacy rates and an index of electoral competition (both of which are argued to be correlated with violence and exogenous to property rights) to assess the effects on the endogenous variable, violence. Findings point to a negative effect of violence on the level of security of property rights. Interesting in this study is the introduction of the average extension of coca fields per municipality as control variable, which also show a negative sign on the level of property rights. 
Other related studies have looked into the determinants of coca fields presence in Colombia, or the decision to plant these illicit crops. \citeA{MorenoSanchez:2003jc}  were the first to study this topic in Colombia, through which the authors identified significant positive effects of the coca base price and total eradicated area, and significant negative effects of the level of other agricultural prices and the total area of coca fields in Peru and Bolivia. As elaborated in the introduction, a difficulty with studying this topic arises through the nature and life cycle of illicit coca leaf planting. This persistency would ideally require modeling of an autoregressive process, which this study lacks. Instead, the model is estimated using lags of the explanatory variables and current levels of the dependent variable. 

A different approach is chosen in the study by \citeA{Ibanez:2010ve}, who looks more closely to the actual decision of farmers to grow coca crops in Colombia. Although considering many control variables such as eradication programs, moral standards, participation in communitarian programs, as well as others, the main result shows that the decision to cultivate coca crops is based on the profit difference compared to other legal activities. Other determinants that have a positive effect on the decision to plant the illicit coca crops are the proportion of coca fields, the number of years cultivating coca crops, and the number of hectares per landowner. 
Some studies have persevered on this difficult topic of the determinants of illicit coca crop fields, with a prime focus on the relationship between violence and illegal armed groups. These studies all incorporate a large number of potential control variables, of which many show a significant effect on coca crop presence. \citeA{R:2005tv}  find a positive and significant effect of violence on coca crops, controlled for many variables including land controls, economic controls, and social controls. However, as brought forward by many studies \cite{Velez:2001vb,Diaz:2004te,Angrist:2008uj,Fernandez:2010wy}, one has to be very cautious with the double causality relationship issue around violence and topics such as the presence of illicit crops, and property rights. It remains difficult to determine in which direction the causal relationship flows, and even more difficult, to disentangle the direction after both factors are present. It could be the case that violence was used by armed groups to pursue farmers to grow illicit coca crops. However, once these crops have been planted, a viscous circle could arise between maintaining the fields and violence. When such a pattern is established, it remains difficult to whether violence is caused by the presence of illicit crops or the other way around, for which one possibility would be to find clever instruments.  In order to tackle the issue, \citeA{Diaz:2004te} use matching estimators to disentangle the direction of the relationship between the territorial expansion of illegal armed groups and violence. Apart from the common control variables, these authors do find a positive and significant effect of illegal armed group activities and coca crop field presence. However, handling the endogeniety issue through matching estimators does come at a price. The matching estimators approach includes a conditional bias term, which is in general not consistent unless particular bias-correction procedures are specified, which are absent in this study \cite{Abadie:2006vj}.

Other studies have reversed this relationship by analyzing the effects the presence of coca crop fields can have on a wide variety of economic outcomes. By looking in the other dirtection, \citeA{Angrist:2008uj} find self-employment income as well as homicide rates to increase significantly in coca growing regions in Colombia. Also child labor and schooling can be effected by this illicit economy. \citeA{Dammert:ue}  uses a difference in difference approach to estimate these effects in Peru, which shows that the presence of these coca fields does seem to suggest a rise in child. However, they shift of the coca production from Peru to Colombia did not have an effect on the enrollment rates in Peru. More in line with the studies on violence and the presence of armed groups in Colombia, \citeA{Velez:2001vb} focuses on the determinants of the territorial expansion of illegal left-wing guerrillas in Colombia, which the author finds to be influenced positively by the presence of the coca crops.

Although all of the studies above do touch upon the topics of the current study partly, they do show the niche that still exists to look further into the role of the property rights regime in a country and its effects on illicit behavior, controlled for the widely apparent violence in Colombia. Furthermore, disentangling all three variables and the causal directions has never been attempted in one coherent study. To put this in a larger perspective, economic institutions such as a certain property rights regime or climate experienced in practice by individuals (i.e. the security of the rights, and being able to execute the rights) could both have large effects on positive behavior such as increasing investments and productivity, but also especially have negative implications when these institutions are absent. A slight derivation from several studies presented above would be to argue that this would block people from investing in their future, causing inactive behavior. We aim to take this argument of \citeA{Acemoglu:2012wb} one step further, by arguing that the absence of certain institutions does not only cause inactive investment behavior of certain individuals, but simply active behavior of others on a rather different path. This active behavior could be by individuals or groups moving in from other areas who are already showing illicit behavior elsewhere, and who derive their newly obtained power from the absence of central authority. Or, it could be argued that local people do not stop to invest in their future at all. These individuals simply keep on investing their resources in a bright future, switching their investment to other opportunities but nonetheless maximizing their utility, although this might not be an investment in a legal activity or an activity which would be held legal by the one�s in power (i.e. the local or central government). Furthermore, it is exactly the one�s in power who cannot provide any security or services to their citizens in the first place. Examples of this are indeed illicit coca crop growing, but also illegal lumbering in a tropical rainforest, blood diamonds, and opium production in Afghanistan  \cite{Angrist:2008uj}. Therefore, the type of economic institutions installed by the one�s in power over the political institutions could evoke behavior what would be commonly held as illicit behavior by the large community or just the one�s in power. This argument does not only hold for the local farmers switching their farming activities to coca crops, but also for paramilitary groups. These groups were founded because the government could not protect them from the violence of the left-wing guerilla groups, after which these groups moved into the coca crop business exactly because of the absence of authority of the government. In the following chapter, this hypothesis will be further decomposed. 


\section{Empirical framework}
\label{identification}

As we describe in sections \ref{intro} and \ref{previous_res}, there is  evidence pointing at a potential relationship between property rights over land and the presence of illicit coca fields. Specifically, recent studies suggest that mechanisms designed to increase households' income from legal activities can have negative effects on the extension of coca leaf plantations \cite{Grossman:2008wh,Ibanez:2010ve,Mejia-Restrepo_2008,MorenoSanchez:2003jc}. Moreover, previous literature also shows that more secure forms of property rights increase wages, productivity, investment and human capital in rural households, and reduce social tensions that subsequently generate violence and breeding grounds for illegal activities \cite{Andre:1998tu,Besley:1995ti,Besley:2000ty,Binswanger:1995td,Collier-etal_2001,Collier-Hoeffler_2004,Deiniger_2003,Deininger:2006ue,Deininger:2006ue,Deininger:2008uq,Deininger:2009tp,Fernandez:2010wy,Velasquez_2008}. 

Likewise, we argue that a causality relationship exists between property rights and the presence of coca leaf plantations. On the one hand, property rights over land are not a volatile economic institution and thus they are persistent over time subject to structural reforms implemented by governments, which results in a subsequent relationship with the initial conditions \cite{Acemoglu:2001tk,Ibanez:2010tr}. On the other hand, the proliferation of coca fields was fueled by the worldwide boom of drug use in the second part of the last century and a reallocation of coca fields among producing countries. Therefore, as this shift in production is exogenous, a positive effect may imply that weaker structures of property rights create the conditions for an expansion of illicit coca leaf plantations.

Furthermore, most studies on the determinants of coca plantations have used cross-sectional or time-series specifications as data on the subject has been limited \cite{R:2005tv,MorenoSanchez:2003jc,Ibanez:2010ve,Diaz:2004te}. However, programs implemented jointly by the United Nations and governments in producing countries have provided additional information that allows to use the time dimension through different individual regions and improve this specification with a panel data approach. Likewise, there is a particular characteristic in the coca harvesting process that has not been explicitly addressed by previous studies. As we point out in section \ref{intro}, coca bushes are perennial and reach full maturity between 12 and 24 months after sowing the seeds\cite{Bray:1983td,USDJ:1991vy,Hanna:1974wd,Matteucci:1996vt,MorenoSanchez:2003jc,Riley:1993ta}. This renders a dynamic nature in the presence of coca fields and its consideration allows for a better understanding of the dynamics of adjustment. Furthermore, failing to control for persistence in the presence of coca leaf plantations raises a source of omitted variables bias.

Additionally, these studies coincide regarding the groups of controls included when studying either the presence of coca plantations, or further outcomes explained by illicit coca crops. A common factor is the use of geographic, political and social controls along with characteristics of the structure of land tenure \cite{Angrist:2008uj,Dammert:ue,Diaz:2004te,Ibanez:2010ve,MorenoSanchez:2003jc,R:2005tv,Velez:2001vb}. Furthermore, the literature points out a double causality relationship between the presence of illicit coca plantations and violence \cite{Angrist:2008uj,Diaz:2004te,Fergusson:2012tx,Velasquez_2008,Fernandez:2010wy,Velez:2001vb}.

In sections \ref{data} and \ref{descriptive}, we describe the dataset we use in this research. All sets of controls follow directly from previous studies on the subject as they correspond to the same analytical framework, albeit adjusted to a municipal level. Furthermore, we explain the details of the econometric model in section \ref{strategy} in which we introduce a linear specification as in \citeA{MorenoSanchez:2003jc} although in a panel data set up, taking into consideration the dynamic nature in the presence of coca leaf plantations and endogeneity issues regarding violence. Results of the estimations are in section \ref{results}. Finally, in section \ref{causalitytests}, we perform Granger causality tests following \citeA{Granger:2003ts}  and \citeA{HoltzEakin:1988wo}.

\subsection{Data}
\label{data}

We use a 10 years date set for 


We use a dataset covering years 2000 to 2008 for 892 municipalities in Colombia to carry out this study. Data regarding the presence of coca fields is provided by the United Nations Office on Drugs and Crime --UNODC--, which collects the net area used for coca cultivation in each municipality of Colombia. Coca fields change mainly due to new plantations, abandonment and reactivation of abandoned fields and eradication programs, and therefore a cut-off date at the end of each year is used in the estimation of the total area under coca cultivation. Furthermore, there are sources of measurement errors regarding the methodology used by UNODC in estimating these areas that need to be highlighted. On the one hand, albeit satellite images are corrected for cloud cover, dates of acquisition and other potential limitations, mountain slopes and recognition of abandoned fields imply further difficulties. On the other hand, the cut-off date at the end of the year rules out short term coca fields that may have produced throughout the year. These drawbacks are partially handled with auxiliary information from the Colombian Government and verification overflights \cite{UNODC:2011tf}.

Moreover, in this study we follow \citeA{Acemoglu:2001tk} ) in measuring institutions with a proxy variable relying on the strength of property rights. In particular, we use an informality index of property rights over land constructed by \citeA{Ibanez:2010tr} using data from the cadastral database of the Geographical Institute Agust\'{i}n Codazzi. The construction of this index is based upon two different land ownership categories defined in Colombia, namely \emph{formal property} and \emph{informal property}\footnote{The definition of \emph{formal property} and \emph{informal property} in Colombian legislation parallels the concepts of \emph{de jure} and \emph{de facto} property rights described by \citeA{Schlager:1992vw}. This is, \emph{formal property} is given full lawful recognition by legal instrumentalities whereas \emph{informal property} originates as a claim by land users}. Ownership is said to be \emph{formal} when it involves the title registration and cadastral identification of the property and it is \emph{informal} when there is only the cadastral identification. \emph{Formal} ownership is the most secure form of property rights over land\footnote{Land formalization programs led by the Ministry of Agriculture and the National Institute for Rural Development aim mainly at issuing registry titles to owners of \emph{informal properties}}. Furthermore, \citeA{Ibanez:2010tr} calculated a corrected cadastral area in each municipality by filtering out of the dataset all non-private properties such as indigenous and Afro-Colombian territories, state owned land and natural reserves, and constructed the index as the proportion of \emph{informal property} out of the corrected cadastral area.

\[
\text{\emph{Informality index land property}}=\frac{\text{\emph{Area of informal properties}}}{\text{\emph{Corrected cadastral area}}}
\]

The bureaucratic procedure for updating cadastral databases with information of legal titles issued by registry offices yields one limitation of the informality index of property rights. This is important as timing in updates varies across municipalities and within areas in a municipality as well. This source of measurement error cannot be eliminated in the scope of this study and therefore the conclusions we derive from this research have to be interpreted considering this limitation.

Likewise, we also consider an alternative measure for institutions. \citeA{Acemoglu:2005ua} provide a theoretical background for the relationship among economic and political institutions. Therefore, we use the municipal development index as a proxy variable to measure political institutions. The municipal development index is constructed by the National Planning Department each year and it is intended to be a measure of general performance of the municipal government in social and financial aspects. It considers social variables such as energy, water and sewerage coverage, poverty, literacy rates and schooling coverage, and financial variables such as tax revenue.

Furthermore, we use the homicide rate per 100,000 inhabitants as a proxy variable for violence. This data is provided by the Presidential Program on Human Rights and International Humanitarian Law, which collects the number of homicides per municipality in Colombia.

Moreover, we consider four sets of additional controls in the dataset. First, we use a time-invariant geographic vector with data on altitude, distance to the nearest main national market\footnote{The main national markets in Colombia are Bogot\'{a}, Medell\'{i}n, Cali and Barranquilla, which concentrate 29\% of the total population as of 2010}, distance to the capital of the department, suitability of land for farming and soil erosion. This data is provided by the Geographic Institute Agust\'{i}n Codazzi.

Second, we use a political vector with information regarding state action carried out in each municipality. In particular, it has data for public per capita expenditures in education, health and justice and the number of agricultural loans per 1,000 inhabitants. Information for public expenditures is provided by the National Planning Department and information on agricultural loans by the Banco Agrario de Colombia, a state owned bank focused on rural development and agricultural banking.

Third, we use a vector of two land variables. The first component is the land quality Gini index, which is a Gini index based upon the minimum number of hectares a rural household needs in order to produce enough resources for long-term wellbeing\footnote{The number of hectares a rural household needs in order to produce enough resources for long-term wellbeing varies from one municipality to another. It is determined by the National Institute for Rural Development according to environmental, social and economic characteristics of each municipality}. A second component is the number of hectares per landowner. Data to construct these variables is provided by the Geographic Institute Agust\'{i}n Codazzi.

Finally, we consider a vector of social variables. Its first component is health coverage, which is provided by the Minister of Health and Social Protection. Up to 2004, the Minister reported health coverage for people living with insufficient basic needs. From 2005 on, the report is for health coverage of the subsidized health care system. The second component is the proportion of people living in rural areas in each municipality which is constructed with data from the National Department of Statistics.

\subsection{Descriptive statistics}
\label{descriptive}

Table~\ref{ds_tvar} presents average summary statistics for time-varying variables per year. The average yearly proportion of coca fields per 1,000 hectares declines up to year 2002 when it stabilizes with few variations per year between 2002 and 2008. This can be seen not only in the national average but in the standard deviation which also stabilized from year 2002 on. Moreover, the average informality index in property rights over land shows a yearly small reduction for the period covered. This is of no surprise, since the formalization process is progressive and slow and it is not expected for a property with a legal title to suddenly lose it. However, there is an increase between the years 2004 and 2005 which may be due to actualization processes in cadastral databases that change total areas in municipalities as a result of the application of improved measurement techniques.

Moreover, the average municipal development index shows an upward trend for the period 2000 to 2008 with exceptions for the years 2003 and 2007. These two years correspond to the last year in office for current municipal incumbents which may provide a political explanation for this phenomena. Furthermore, the average homicide rate shows a downward trend with high levels of variability among municipalities. Additionally, a remarked difference can be identified for two different periods as the homicide rate ranged between 52 and 60 for the period 2000-2003 to a range between 33 and 40 homicides per 100,000 inhabitants for the period 2005-2008 with a middle point at 44 in 2004.

Average public per capita expenditures in education and justice present no trend for the period 2000-2008 and a high variability among municipalities can be identified in the standard deviations. However, average public per capita expenditures in health show a remarked increase for the last four years. Moreover, the average number of agricultural loans per 1,000 inhabitants presents no trend for the first half of the period covered with a similar situation in the second half. Nevertheless, it ranges between higher levels in the last years.

Moreover, the average land quality Gini index and the average number of hectares per landowner exhibit few changes in the period 2000 to 2008. This is an expected situation since changes in the structure of land tenure are those of long-term, usually as a result of land reforms. Finally, average health coverage presents an upward trend for the period covered in the dataset. However, it is important to notice that this upward trend splits for the periods 2000-2004 and 2005-2008, this is due to the change in measurement methodology followed by the Minister of Health and Social Protection. Finally, the proportion of people living in rural areas presents a downward trend for the years 2000-2004 followed by an increase in the year 2005 and no trend for the rest of the period. An important factor in changes of this rurality index is the forced displacement caused in rural areas of Colombia because of violence, which resulted in massive migrations from rural to urban areas in the first part the decade. The national census carried out in 2005 may be the cause of the sudden increase that is identified this year. Finally, heterogeneity across municipalities can be seen in all time-varying variables by comparing means and standard deviations.

Furthermore, table~\ref{ds_tinv} presents summary statistics for time-invariant geographic characteristics. These variables exhibit the same behavior as all time-varying variables regarding heterogeneity across municipalities.

Table \ref{ttests} summarizes means and standard deviations for all variables taking averages over the period of study. The first column presents statistics for municipalities that never had illicit coca plantations, the second column for those municipalities that ever had illicit crops and the third column presents the respective differences. Furthermore, the difference in the informality index is negative and statistically significant, pointing at a higher level of informality in municipalities that ever had illicit crops. This is also true for the homicide rates. On the other hand, the municipal development index shows the converse situation, this is, higher levels of municipal development in municipalities that never had illicit coca plantations.

Moreover, table~\ref{ds_trans} shows transition probabilities on the presence of coca fields. It shows that $97.77\%$ of the municipalities where there were no coca fields at a given period remained without them for the next period. Respectively, $85.98\%$ of the municipalities where there were coca fields at one period remained holding coca crops the next period. Furthermore, table~\ref{ds_tabul} presents tabulations on the presence of illicit  fields. It shows that on average $13.91\%$ of the municipalities in Colombia had illicit coca fields and $21.64\%$ ever had them. Likewise, $64.31\%$ of the municipalities that ever had coca fields, had them during all the periods subject of study.

Additionally, table~\ref{ds_corr} presents correlation coefficients for relevant variables. Three important facts to note are first, the proportion of coca fields per 1,000 inhabitants has a positive and significant correlation with the informality index in property rights over land as well as with the homicide rate. Conversely, it has a negative and significant correlation with the municipal development index. Second, the homicide rate has no significant correlation with the informality index and a negative and significant correlation with the municipal development index. And finally, the informality index has positive or negative significant correlations with all other controls except for the homicide rate and public per capita expenditures in education and justice.

Furthermore, figure~\ref{ds_map} presents a map of Colombia at the municipal level describing the presence of coca fields and the informality index of property rights. It shows quartiles of the average informality index and whether there were coca fields in a municipality in any year for the period 2000-2008. Notably, the south-west and central regions, which are on the fourth quartile of the informality index hold most of the illicit coca fields identified during the period subject of study, and the converse applies for regions in lower quartiles of the informality index. This relationship can be seen in figure~\ref{ds_lfit} as well, which presents a linear fit between the proportion of illicit coca fields per 1,000 hectares and 100 quantiles of the informality index of property rights over land. 

Finally, table~\ref{ds_xtsum} describes between and within variations for all time-varying variables in the dataset. The proportion of coca fields per 1,000 hectares presents similar variations between municipalities and within each municipality although the variation within each municipality is higher. Moreover, for the informality index of property rights over land, the variation within each municipality was notably smaller than the variation between municipalities. An expected situation due to the slow progress in formalization programs.

\subsection{Estimation strategy}
\label{strategy}

In this paper we study the determinants of the presence of illicit coca plantations. In particular, we focus on the effects of the informality index of property rights over land. As we argue in section \ref{intro}, this relationship is characterized by several relevant facts. First, the process is dynamic in the sense that past realizations of the proportion of area in a municipality with illicit coca fields may affect the current one since coca plantations are perennial and reach full maturity between 12 and 24 months after sowing the seed\cite{Bray:1983td,GAO:2008uo,Hanna:1974wd,Matteucci:1996vt,MorenoSanchez:2003jc,Riley:1993ta}.    Failing to account for characteristics of persistence in the presence of coca fields may yield a source of omitted variable bias. Second, heterogeneity among municipalities is an indicator of the existence of fixed individual effects, which argues against cross-sectional regressions and favors a panel data set up that allows to use the variation over time to estimate the parameters and rule out other sources of potential omitted variables bias. Third, there is empirical evidence suggesting that one control, violence, holds a double causality relationship with the presence of illicit coca fields \cite{Angrist:2008uj,Diaz:2004te,Fernandez:2010wy,Velez:2001vb}. Fourth, there is no evidence to assume that idiosyncratic disturbances do not have patterns of heteroskedasticity and serial correlation. Finally, the data available is that of small $T$ and large $N$ with 9 time periods covering years 2000 to 2008 and 892 municipalities.

Given these considerations, we follow \citeA{HoltzEakin:1988wo} , \citeA{Arellano:1991wr}, \citeA{Arellano:1995tj} and \citeA{Blundell:1998vr} and propose a linear specification of the following form:
\begin{equation}
\label{model}
y_{it}=\alpha y_{it-1}+\delta p_{it} + \gamma h_{it} + \mathbf{x'}_{it}\beta + \varepsilon_{it}
\end{equation}

Where the subindex $i$ stands for the municipality and $t$ for the time period. The dependent variable $y$ represents the proportion of illicit coca fields and its first lag is introduced as explanatory variable, $p$ is the informality index of property rights over land, $h$ is the homicide rate, a proxy variable for violence and therefore it is known to be endogenous. Finally, $\mathbf{x}$ is a vector of geographic, political, land and social controls. The parameter of interest is $\delta$ and it is expected to be positive.

Furthermore, the disturbance term has two orthogonal components. A fixed effect $\mu _{i}$ and an idiosyncratic shock $v_{it}$, such that:
\begin{equation}
\label{ass_errors}
\begin{aligned}
\varepsilon_{it}&=\mu _{i}+v_{it}\\
E\left[\mu_{i}\right]&=E\left[v_{i}\right]=E\left[\mu_{i}v_i\right]=0
\end{aligned}
\end{equation}

As it is pointed out by \citeA{Baltagi:2008uo}, beyond the consideration of  endogeneity issues with $h$, the introduction of a lag of the dependent variable in (\ref{model}) yields a situation known as  \emph{dynamic panel bias}. For instance, since $y_{it}$ is a function of $\mu_{i}$, $y_{it-1}$ is also a function of $\mu_{i}$ and thus it is correlated with the error term. This implies that the OLS estimator is biased and inconsistent \cite{Sevestre:1985wf}. Furthermore, although the within transformation\footnote{The within transformation takes the levels equation $y_{it}=\alpha y_{it-1} + \beta x_{it} + \mu_i+v_{it}$ and subtracts out the average over time $\bar y_{i.}=\alpha \bar y_{i.-1} + \beta \bar x_{i.}+\mu_i +\bar v_{i.}$. The resulting model is of the form $(y_{it}-\bar y_{i.})=\alpha (y_{it-1}-\bar y_{i.-1})+ \beta (x_{it}-\bar x_{i.}) +(v_{it}-\bar v_{i.})$} for the Fixed Effects estimator removes $\mu_{i}$, the idiosyncratic shock $v_{it-1}$ is correlated with $y_{it-1}$ and therefore in the transformed model $(y_{it-1}-\bar y_{i.-1})$ will be correlated with $(v_{it}-\bar v_{i.})$. This renders the Fixed Effects estimator to be biased and inconsistent as well \cite{Nickell:1981uo}.

Two approaches that follow the generalized-method-of-moments developed by \cite{Hansen:1982ue}  to overcome the \emph{dynamic panel bias} are the Difference GMM estimator, proposed by \citeA{HoltzEakin:1988wo} and \citeA{Arellano:1991wr} in which the data is converted using the first-difference transformation\footnote{The first-difference transformation is performed by multiplying $y_{it}=\alpha y_{it-1} + \beta x_{it} + \mu_i+v_{it}$ by $\mathbf{I}_N \otimes \mathbf{M}_{\Delta}$ with $\mathbf{I}_N$ being an identity matrix of order $N$ and $\mathbf{M}_{\Delta}$ a matrix with diagonal of $-1s$ and a sub-diagonal of $1s$ to the right. The transformation yields $\Delta y_{it}=\alpha \Delta y_{it-1} + \beta \Delta x_{it} +\Delta v_{it}$. Although the term $y_{it-1}$ in $\Delta y_{it-1}$ is still correlated with $v_{it-1}$ in $\Delta v_{it}$, longer lags of the regressors are orthogonal to the error term and can be used as instruments (see \cite{Roodman:2009tc}) for additional details)} in order to remove the fixed effects, and the System GMM estimator, developed by \citeA{Arellano:1995tj} and \citeA{Blundell:1998vr}, where instead of transforming the regressors, the instruments are transformed so as to make them exogenous to the fixed effects. Specifically, System GMM not only estimates the equations in first-differences but it also estimates the equations in levels, instrumented with lagged first-differences of the corresponding variables. Therefore, the first set of moment conditions exploited by the System GMM estimator are those of the first-difference transformation, namely\footnote{We do not include orthogonality conditions corresponding to predetermined or weakly exogenous variables as no covariate is considered as such in the model. See the appendix for a further explanation on orthogonality conditions regarding predetermined variables}:
\begin{equation}
\label{momentsDiff}
\begin{aligned}
E\left[ y_{it-s}\Delta \varepsilon_{it}\right]&=0 \text{; for $t=3,...,T$, $2\leq s \leq t-1$}\\
E\left[w_{is}\Delta \varepsilon_{it}\right]&=0 \text{; for $t=3,...,T$, $1\leq s \leq T$ and any exogenous covariate $w$}\\
E\left[w_{it-s}\Delta \varepsilon_{it}\right]&=0 \text{; for $t=3,...,T$, $2\leq s \leq t-1$ and any endogenous covariate $w$}
\end{aligned}
\end{equation}

Additionally, an identifying assumption is required to make the first-differences of the explanatory variables exogenous to the fixed effects even if the fixed effects are correlated with the variables in levels \cite{Arellano:1995tj,Blundell:1998vr}. This is\footnote{In order to exploit these additional moment conditions the dataset is stacked with both the transformed and the untransformed observations. This is done by multiplying the original dataset by $\mathbf{M}=\left(\mathbf{M}_{\Delta}\;\;\mathbf{I}\right)'$ which yields an augmented dataset of the form $\mathbf{X}_{i}=\left(\mathbf{\Delta x}_{i}\;\;\mathbf{x}_i\right)', \mathbf{Y}_{i}=\left(\mathbf{\Delta y}_{i}\;\;\mathbf{y}_i\right)'$ for each individual $i$. Information for all covariates is in matrix $\mathbf{x}$ and $\mathbf{M}_{\Delta}$ is the matrix used in the first-difference transformation, a matrix with diagonal of $-1s$ and a sub-diagonal of $1s$ to the right (see \cite{Roodman:2009tc} for further details)}:
\begin{equation}
\label{momentsSys}
\begin{aligned}
E\left[\Delta y_{it-1}\varepsilon_{it}\right]&=0 \text{; for $t=3,...,T$}\\
E\left[\Delta w_{it}\varepsilon_{it}\right]&=0 \text{; for $t=2,...,T$ and any exogenous covariate $w$}\\
E\left[\Delta w_{it-1}\varepsilon_{it}\right]&=0 \text{; for $t=3,...,T$ and any endogenous covariate $w$}
\end{aligned}
\end{equation}

Moreover, if these additional moment conditions are valid, the System GMM estimator outperforms the efficiency of the Difference GMM counterpart, specially when explanatory variables are persistent or the number of time series observations is small relative to the number of individuals\cite{Blundell:1998vr,Blundell:2000wu}. 

Furthermore, although both estimators are consistent and use the first-difference transformation to remove sources of potential omitted variables bias, the System GMM estimator is preferable in the proposed framework for a number of reasons. First, as the additional moment conditions given by (\ref{momentsSys}) are overidentifying restrictions, they can be assessed empirically by using standard tests of overidentification and serial correlation and therefore achieve a substantial gain in efficiency \cite{Blundell:1998vr}. Second, since the specification considers a set of time-invariant geographic controls, these explanatory variables would disappear when performing the first-difference transformation required for the Difference GMM estimator. By using the System GMM estimator, time-invariant covariates can be considered in the model. Therefore, we use System GMM to estimate the parameters in (\ref{model}).

However, when implementing the System GMM estimator, the number of instruments grows exponentially with the number of time periods in the panel and thus the gain in efficiency due to further orthogonality conditions comes with additional costs. The proliferation of instruments causes three difficulties. First, a bias in coefficient estimates because instruments over-fit instrumented variables \cite{Ziliak:1997vg,Windmeijer:2005vw}. Second, an increase in the likelihood of false-positive results in the Hansen test of overidentification restrictions \cite{Andersen:1996ui,Bowsher:2002vj}. Finally, a downward bias in coefficient standard errors when performing the two-step GMM estimation \cite{Windmeijer:2005vw}. In order to overcome these difficulties, we reduce the instrument count by using a collapsed instruments matrix \cite{Christiaensen:2011wq,Heid:2012ut,Beck:2004ux,Roodman:2009tc}. Furthermore, we employ \citeA{Windmeijer:2005vw}  finite-sample correction for standard errors. For additional specificities on the empirical strategy we provide a detailed explanation of the System GMM estimator in the Appendix.

\subsection{Results}
\label{results}

We estimate three different characterizations of the model. In the most parsimonious version we do not consider the homicide rate as explanatory variable. This allows to observe the effect of informality without the influence of violence and serves as a baseline analysis on the relationship between property rights and the presence of illicit coca fields. A second characterization introduces violence. However, even though \citeA{Angrist:2008uj}, \citeA{Fernandez:2010wy}, \citeA{Diaz:2004te} and \citeA{Velez:2001vb} provide evidence to consider endogeneity concerns in the relationship between coca leaf plantations and violence, we do not control for endogeneity. The final characterization controls for endogeneity matters regarding the homicide rate. 

As the lag of the presence of illicit coca fields is endogenous by construction, it is instrumented with (collapsed) lags 2 and further in all estimations. Furthermore, the homicide rate is handled in the same manner when instrumented. The usefulness of these instruments is supported by overidentification and serial correlation tests, which we discuss in detail in sections \ref{overid} and \ref{auto}. Likewise, as autocorrelation tests and robust estimates of the standard errors assume no contemporaneous correlation across individuals, all estimates consider time dummies.

Table~\ref{results1} shows the results for the first characterization of the model. Three relevant facts are important to highlight. First, the effect of the informality index in the presence of coca fields per 1,000 hectares is positive and highly significant in all regressions. Second, as more controls are considered in the model there is a moderate reduction in the magnitude of the coefficient of the informality index. Furthermore, the geographic controls seem to have the strongest decreasing effect since after their introduction the coefficient hardly diminishes. This stands for a strong relationship between the informality index in property rights and the presence of coca plantations. Finally, the coefficient of the lag of the dependent variable is smaller than $1$ in all regressions, a condition pointed out by \citeA{Blundell:1998vr} so that the dynamic process converges. In addition, the coefficient is positive and highly significant. This is relevant in assessing the specification of the model.

Table~\ref{results2} presents the results for the second characterization of the model. A first relevant fact is that the coefficient of the informality index in property rights barely changes compared to the parsimonious characterization. It remains both positive and highly significant in all regressions and the magnitudes are mildly smaller. As in the previous estimates, the strongest decreasing effect in the magnitude of the coefficient is brought about by the introduction of geographic controls. Likewise, the coefficient of the homicide rate is also positive and significant in all regressions. Finally, the coefficient of the lag of the dependent variable remains smaller than $1$.

Table~\ref{results3} shows the results for the model where endogeneity issues concerning the homicide rate are handled. Several facts are important to mention regarding this characterization. First, results for the informality index are robust to the instrumentation of violence as the coefficient is still positive and highly significant in all regressions. Second, the magnitude of the coefficient of the informality index increases and reaches similar but higher levels than those of the first characterization. Third, as in the previous two cases, the introduction of geographic controls yields the most important reduction in the magnitude of the coefficient of the informality index. Likewise, the consideration of further controls causes less decreasing effect. Fourth, the coefficient of the lagged value of the dependent variable presents similar values as in previous characterizations of the model. It is a positive and highly significant coefficient smaller than $1$. In addition, the effect of violence in the presence of coca fields is higher when controlling for endogeneity issues and the coefficient is still positive and significant in all regressions. Finally, the usefulness of instruments for the homicide rate and lagged value of the dependent variable is supported by overidentification and serial correlation tests. On the other hand, column $(5)$ of this table renders significant coefficients for 9 of the explanatory variables considered in the specification. we now focus on the interpretation of signs and magnitudes of the coefficients for relevant variables.

As the subject of study in this paper, the coefficient of the informality index in property rights was expected to be positive, a result that is confirmed in all estimates. This validates the hypothesis that weaker institutions, measured as weaker structures of property rights over land have a positive effect on the levels of coca leaf plantations. As the informality index of property rights and the dependent variable are proportions, the coefficient of $0.849$ is to be interpreted as a positive effect of $0.849$ percent in coca fields per $1,000$ hectares as informal properties in a municipality increase by $1$ percent. 

Furthermore, the coefficient of the homicide rate is positive and significant in all regressions. This finding confirms the results from other empirical studies that found a positive effect of violence on the presence of coca fields \cite{Angrist:2008uj,Diaz:2004te,Fernandez:2010wy,Velez:2001vb}. As the homicide rate can be read as a percentage as well, the interpretation of the magnitude of the coefficient follows directly from the previous case. Results show a positive effect of $0.007$ percent in coca fields per $1,000$ hectares as the homicide rate per $100,000$ inhabitants increases by $0.01$. 

On the other hand, geographic characteristics as distance to the nearest main national market and distance to the capital of the department have positive and significant effects on the presence of coca leaf plantations. This is an expected outcome as far-off municipalities are more likely to be subject of illegal activities and confirms the findings by \citeA{Diaz:2004te} concerning the effect of the distance to the nearest main national market on the presence of illicit plantations.

Furthermore, the number of agricultural loans per 1,000 inhabitants shows a negative and significant effect on the presence of illicit coca plantations which confirms the findings by \citeA{R:2005tv}. Agricultural loans are aimed at promoting legal farming and therefore provide an alternative source of income to peasants that otherwise would be growing coca fields. \citeA{Ibanez:2010ve} finds a similar relationship, as her results show that the decision of cultivating coca relies on the profit difference between coca growing and other legal activities.

Likewise, the number of hectares per landowner, a measure of the average plot size per municipality, shows a negative and significant effect on the presence of illicit coca plantations. This also confirms the findings by \citeA{Ibanez:2010ve}  on the effect of the number of hectares per landowner on coca growing decisions in rural households.

Moreover, health coverage shows a negative and significant effect on the proportion of illicit coca plantations. This is an expected result as similar relationships for other social characteristics such as educational coverage and quality-of-life index are in \citeA{Diaz:2004te} and \citeA{R:2005tv}. On the other hand, the proportion of people living in rural areas shows positive and significant effects on the proportion of illicit coca fields confirming the same result in \citeA{R:2005tv}.

Finally, the one year lag of the dependent variable presents a positive and significant effect on its current value. There are two important facts regarding this result. First, the maturity of coca plantations is reached between 12 and 24 months after the seeds are sown and therefore it is likely for a coca growing peasant to pursue further periods of harvest at least for the year following the sowing time. Second, it confirms the findings by \citeA{Ibanez:2010ve} on the effect of years cultivating coca on the coca growing decisions in rural households. Moreover, the coefficient is smaller than $1$ which is a necessary condition for the process to converge \cite{Blundell:1998vr}.

\subsection{Causality tests}
\label{causalitytests}

We study whether there exists a causal relationship between informality in property rights over land and the presence of illicit coca leaf plantations. we do so by performing standard Granger causality tests which were originally designed for time-series but have recently been extended to panel data analysis \cite{Granger:1969wx,Granger:2003ts,HoltzEakin:1988wo}. The causality linkage is studied upon the following two models:
\begin{equation}
\label{causality1}
y_{it}=\alpha y_{it-1}+\sum\limits_{j=0}^{2} \delta_j p_{it-j} + \varepsilon_{it}
\end{equation}
\begin{equation}
\label{causality2}
p_{it}=\delta p_{it-1}+\sum\limits_{j=0}^{2} \alpha_j y_{it-j} + \varepsilon_{it}
\end{equation}

Where, as in section \ref{strategy}, the subindex $i$ stands for the municipality and $t$ for the time period, $y$ represents the proportion of illicit coca fields while $p$ is the informality index of property rights over land. Although these models could be estimated using different consistent estimators as for instance Difference GMM, we estimate the models following a System GMM approach to rule out \emph{dynamic panel bias} and gain a substantial efficiency. Furthermore, the first-difference transformation used in the estimation allows for fixed-individual effects. Therefore, conditions analogous to those given by (\ref{ass_errors}), (\ref{momentsDiff}) and (\ref{momentsSys}) are assumed to be satisfied in both cases. The test of weather informality in property rights does not Granger-cause an increase in coca leaf plantations is a test of the joint hypothesis $\delta_0=\delta_1=\delta_2=0$ after estimating model (\ref{causality1}). The converse relationship is tested in the same way for the model specified in (\ref{causality2}) and the joint hypothesis $\alpha_0=\alpha_1=\alpha_2=0$. we do this with standard $F$-tests.

Table \ref{causalityreg1} presents results for the estimation of model (\ref{causality1}) including one and two lags of the informality index of property rights. In both cases the F test for no Granger causality is rejected which implies that higher levels of the informality index of property rights Granger-cause an increase in coca leaf plantations. Respectively, table \ref{causalityreg2} presents results for the System GMM estimation of model (\ref{causality2}) including one and two lags of the proportion of coca fields per 1,000 hectares. In this case, the F test of no Granger causality is not rejected at any level of significance which yields no causality in this direction. Moreover, the Hansen test for the specification of both models is not rejected in any case. The same is true for the Difference-in-Hansen test, which tests the validity of the additional orthogonality conditions for the System GMM estimation given by the moments analogous to those in (\ref{momentsSys}) for each model. The instruments used for the endogenous variables are also supported by Arellano-Bond tests of serial autocorrelation as (collapsed) lags 2 and further are used as instruments in each case. See the Appendix for further details on the validity of the System GMM estimation.

Although Granger causality does not necessarily implies true causality as third party variables may affect both series, the test provides an insight in the causality linkage between the presence of illicit coca plantations and informality in property rights.

\section{Robustness analysis}

\subsection{Overidentification tests}
\label{overid}

The usefulness of the generalized-method-of-moments procedure requires the instruments to be exogenous. This is true if the set of orthogonality conditions given by (\ref{momentsDiff}) and (\ref{momentsSys}) are fulfilled. Therefore, we investigate the validity of my approach by performing the Hansen test of overidentification restrictions \cite{Hansen:1982ue}. The null hypothesis of the test is a correct specification of the model and the validity of the orthogonality conditions. If either of these two assumptions is questionable the test is rejected.

Tables \ref{results1}, \ref{results2} and \ref{results3} report the $p$-value for the Hansen test of overidentification restrictions. Interestingly, the null hypothesis of correct specification and validity of the orthogonality conditions is never rejected. Moreover, the latter characterization of the model in which we control for endogeneity concerns with the homicide rate shows a remarkable improvement in the results of the test, which stands for a refined specification. Likewise, as it can be seen in table \ref{resultsR_pol}, the use of political institutions as an alternative measure for institutions is robust to this analysis as well.

\subsubsection*{Testing for subsets of instruments}

In addition to the Hansen test of overidentification restrictions where the full set of orthogonality conditions is considered, we use the Difference-in-Hansen test for subsets of instruments \cite{Newey:1985tf}. This test computes the increase in the statistic of the Hansen test whenever a given subset of suspect instruments is added to the specification, i.e., the statistic of the Difference-in-Hansen test is the difference between the statistic of the Hansen test with the full set of instruments and the statistic of the Hansen test without suspect instruments. The null hypothesis of the test is the joint validity of the orthogonality conditions corresponding to a determined subset of suspect instruments. Furthermore, we perform the Difference-in-Hansen test for the System GMM suspect instruments given by (\ref{momentsSys}). This is particularly important in assessing the additional conditions required for the use of the System GMM estimator as this estimator is more efficient than its Difference GMM counterpart only if the orthogonality conditions in (\ref{momentsSys}) are valid.

Results for the $p$-value of the Difference-in-Hansen test are reported in tables \ref{results1}, \ref{results2} and \ref{results3} for the model with the informality index, and in table \ref{resultsR_pol} for the model with the municipal development index. The null hypothesis of joint validity of the orthogonality conditions given by (\ref{momentsSys}) is not rejected in any case. Additionally, as in the case of the Hansen test of overidentification restrictions, results of the Difference-in-Hansen test largely improve in the latter characterization of the model in which the homicide rate is instrumented.

\subsection{Autocorrelation tests}
\label{auto}

We also investigate the validity of the instruments by looking for  autocorrelation in the idiosyncratic shocks $v_{it}$. This is important as the presence of serial correlation renders some specific lags invalid as instruments \cite{Arellano:1991wr}. Since the full disturbance $\varepsilon_{it}$ is presumed to be serially correlated because of the presence of the fixed effects $\mu_{i}$, we use \citeA{Arellano:1991wr} test for serial correlation which is applied to the residuals after the first-difference transformation, namely $\Delta v_{it}$.

Results for the \citeA{Arellano:1991wr} test for serial correlation of order $1$ and $2$ are reported in tables \ref{results1}, \ref{results2} and \ref{results3} for the model with the informality index as a measure of economic institutions, and in table \ref{resultsR_pol} for the model with the municipal development index as a measure of political institutions. The null hypothesis of the tests is no serial correlation of order $1$ and $2$ respectively. Notably, in all results the null hypothesis of no serial correlation of order $1$ is rejected and the converse applies to serial correlation of order $2$. This supports the validity of the specification of the model and usefulness of lags $2$ and further as instruments for endogenous variables \cite{Arellano:1991wr}.

\subsection{Political institutions}

We investigate the robustness of my results using an alternative measure for institutional features. In particular, we follow the theory developed by \citeA{Acemoglu:2010ve} on the dynamic interaction between political and economic institutions and based on this relationship we use a proxy variable for political institutions. The underlying rationale of the theory is that political institutions such as a democratic process to elect the head of the executive determine further economic institutions such as property rights over land.

As we point out in section \ref{data}, the municipal development index is a measure of performance for local governments and is constructed upon results on social and financial matters. Therefore, we argue that a higher municipal development index implies a more efficient set of political institutions at the local level. Moreover, following a similar postulate as that in the beginning of section \ref{identification} in the case of property rights over land, political institutions persist over time subject to structural reforms implemented by the constituent \cite{Acemoglu:2001tk}. Hence, it follows that political institutions are a potential exogenous determinant of the presence of illicit coca leaf plantations.

In table~\ref{resultsR_pol}, results show a negative and significant effect of the municipal development index on the presence of illicit coca plantations. Additionally, as more controls are considered there is a moderate reduction in the absolute value of the magnitude of the coefficient and the highest decreasing effect is brought about by the introduction of geographic controls. These results confirm the outcome of section \ref{results} so that weaker institutions, whether political or economic, yield higher levels of coca leaf plantations per 1,000 hectares. All regressions correspond to the latter characterization of the model, in which the lag of the dependent variable and the homicide rate are considered endogenous and instrumented with (collapsed) lags 2 and further. Finally, overidentification and serial correlation tests validate the usefulness of the selected instruments.

\subsection{Alternative samples}

We also analyze the robustness of my results by using two alternative subsamples. First, we consider only municipalities located in departments (states) that ever had illicit coca plantations. This allows to study the relationship between property right institutions and coca crops only in those regions that have been characterized for having crops, which may correspond either to environmental or strategical features, as these regions may facilitate for instance transport of coca leafs. Second, we include only municipalities that are located below 1,500 meters, which is the optimal altitude for growing the plant. This second subsample rules out municipalities that are not expected to have illicit crops.

Table \ref{asamples} presents results for the System GMM estimation using both subsamples. The coefficient of the informality index of property rights over land is positive and statistically significant in all estimates, which supports results from section \ref{results}. All regressions correspond to the latter characterization of the model, in which the lag of the dependent variable and the homicide rate are considered endogenous and instrumented with (collapsed) lags 2 and further. Similarly, overidentification and serial correlation tests validate the usefulness of the selected instruments.

\section{Concluding remarks}

In this paper, we explore the relationship between property rights over land and the presence of illicit coca leaf plantations. Previous literature outlines this relationship in two respects. On the one hand, it shows that more secure forms of property rights increase wages, productivity, investment and human capital in rural households, and reduce social tensions that subsequently generate violence and breeding grounds for illegal activities
\cite{Andre:1998tu,Besley:1995ti,Besley:2000ty,Binswanger:1995td,Collier-Hoeffler_2004,Collier-etal_2001,Collier_1999,Deininger:2006ue,Deininger:2008uq,Deininger:2009tp,Fernandez:2010wy,Velasquez_2008}
. On the other hand, it suggest that mechanisms attempting to increase households' income from legal activities can have negative effects on the extension of coca leaf plantations \cite{Grossman:2008wh,Ibanez:2010ve,Mejia-Restrepo_2008,MorenoSanchez:2003jc}. Moreover, we also exploit the exogenous shift in coca production from Bolivia and Peru to Colombia and characteristics of persistence in property rights institutions to imply a causal relationship. This is, weaker structures of property rights create the conditions for an expansion in illicit coca leaf plantations.

Furthermore, given specific characteristics of the coca cultivating process, we specify a dynamic model for which common approaches as the OLS and Fixed Effects estimators are biased and inconsistent \cite{Nickell:1981uo,Sevestre:1985wf}. Therefore, we use the System GMM estimator which allows to control for individual specific effects and to introduce lags of endogenous variables as instruments. Additionally, it performs better than similar estimators in the presence of persistent variables \cite{Arellano:1995tj,Blundell:2000wu,Blundell:1998vr}. Likewise, we perform Granger causality tests to study the causality linkage between property rights and the presence of coca plantations \cite{Granger:1969wx,Granger:2003ts,HoltzEakin:1988wo}.

Results suggest a positive effect between weaker structures of property rights over land and levels of coca leaf plantations. Specifically, we find an effect of $0.849$ percent in coca fields per $1,000$ hectares as informal properties in a municipality increase by $1$ percent. The relation is robust to different specifications, alternative measures for institutions, distinct subsamples and overidentification and serial correlation tests. Moreover, Granger causality tests show that higher levels of the informality index in property rights Granger-cause an increase in coca leaf plantations whereas the converse causality relationship is rejected.

Overall, these results point at the design of counter narcotics policies based on the assignment of more secure forms of property rights over land, as this can have a negative impact on the extension of illicit coca leaf plantations. Interestingly, by implementing such policies, a reduction in the presence of coca plantations may come along with gains in economic performance, and reductions in poverty and inequality which is expected under better institutions. However, although enhancing property rights is in principle feasible for producing countries, its implementation yields further difficulties, particularly in violent regions where control of the state is not complete. Therefore, additional studies on land reform and alternative ways to improve property rights should accompany the implementation of further policies.

Finally, we identify two limitations concerning the data for this study. First, there exists a source of measurement errors for the informality index in property rights and the presence of illicit coca plantations. Both follow respectively from the methodologies used by Colombian cadastral authorities and the United Nations Office on Drugs and Crime. Second, there is a potential inconsistency for certain municipalities regarding data for the informality index and coca crops as estimations for areas under coca cultivation do not discriminate for instance for indigenous and Afro-Colombian territories, state owned land and natural reserves. These lands are not considered when constructing the informality index in property rights over land as more secure forms of property rights are not feasible unless special procedures are implemented to subtract particular properties out of them. This could be controlled by considering spacial meta-data from UNODC estimations and cadastral information. Innovative alternatives to overcome these difficulties and then redo the estimations may provide more robust results.

\newpage

\bibliography{references}
%
\bibliographystyle{apacite}

\newpage
\singlespacing
\appendix

\section*{Maps}

\end{document}